% !TEX encoding = UTF-8
% !TEX TS-program = pdflatex
% !TEX root = ../tesi.tex

%**************************************************************
\chapter{Appendice A - Grado di compatibilità}
\label{cap:appA}
%**************************************************************

Nel caso in esame per dare la possibilità al cliente di modificare alcune componenti del suo PC senza risultare di utilizzare un computer diverso, è calcolato il grado di compatibilità delle due componenti Hardware invece di verificare che siano identiche.\\ 
L'algoritmo si basa sull'idea di assegnare dei pesi alle componenti del PC in base alla probabilità che esse vengano cambiate. Più una componente è fondamentale, più peso avrà. Alla fine dei controlli è calcolato il punteggio ottenuto come somma dei pesi delle parti cambiate. Se il punteggio supera la soglia massima il computer risulta diverso.
\\Per assicurare una buona efficienza di questo metodo sono state scelte molte componenti di un computer, anche non fondamentali. I pesi sono stati attribuiti accuratamente e pensando a possibili situazioni reali.
\\Per motivi di sicurezza le componenti scelte per il controllo non sono esposte in questo documento.





% !TEX encoding = UTF-8
% !TEX TS-program = pdflatex
% !TEX root = ../tesi.tex

%**************************************************************
\chapter{Introduzione}
\label{cap:introduzione}
%**************************************************************
Questo capitolo ha lo scopo di fornire una breve descrizione dell'azienda ospitante, della struttura del documento e delle norme utilizzate per la stesura dello stesso.

%**************************************************************
\section{L'azienda \emph{VISIONEIMPRESA s.r.l.}}

\begin{figure}[!h] 
    \centering 
    \includegraphics[width=0.9\columnwidth]{LogoVision} 
    \caption{Logo dell'azienda VISIONEIMPRESA s.r.l.}
\end{figure}

\emph{VISIONEIMPRESA S.r.l.} è un’azienda nuova, con sede a Pernumia (PD), ma con una storia che parte dal 1981. Da più di trent’anni si occupa d’informatica e nello specifico di applicazioni gestionali.\\
Nei primi anni la loro attività era dedicata ad aziende, enti pubblici, studi professionali e centri elaborazione dati, gestendo la maggior parte delle problematiche informatiche, la progettazione dei sistemi, l’hardware, le reti, i sistemi operativi e il software applicativo. Con il passare degli anni l’azienda ha deciso di specializzarsi, dedicandosi in modo particolare alle aziende. L’esperienza e il know How acquisiti nella gestione aziendale hanno quindi spinto a dedicarsi esclusivamente al software applicativo e ai relativi servizi di implementazione dello stesso.\\
Il Software Gestionale Vision da loro prodotto permette alle piccole e medie aziende italiane di impostare sul sistema informatico la completa organizzazione aziendale per affrontare un futuro sempre più complesso e veloce con il supporto di un sistema informatico, che aiuti l’azienda a prendere decisioni basate su dati precisi.\\
Grazie anche alle certificazioni Microsoft, sia di partnership che di prodotto, l’azienda dimostra di poter fornire ai propri clienti prodotti e servizi di qualità, migliorandoli costantemente.

%**************************************************************
\section{Struttura del documento}

In questo paragrafo è mostrata la struttura del documento per una maggiore comprensione dei contenuti e per permettere al lettore di trovare facilmente le informazioni di suo interesse.
\\
\\
Il secondo capitolo, {\hyperref[cap:descrizione-stage]{Descrizione dello stage}}, descrive nel dettaglio le problematiche e le soluzioni affrontate nell'attività di stage, mostrando una breve analisi dei rischi, la pianificazione del lavoro e gli obiettivi da raggiungere decisi prima dell'inizio delle attività.
\\
Il terzo capitolo, {\hyperref[cap:ambiente-sviluppo]{Ambiente di sviluppo}}, approfondisce gli strumenti utilizzati per realizzare le soluzioni proposte.
\\ 
Il quarto capitolo, {\hyperref[cap:sviluppo-software]{Sviluppo Software}}, illustra dettagliatamente le soluzioni realizzate per soddisfare i requisiti richiesti.
\\ 
Il quinto e ultimo capitolo, {\hyperref[cap:analisi-retrospettiva]{Analisi retrospettiva}}, contiene un'analisi riassuntiva degli obiettivi raggiunti, delle conoscenze acquisite e le conclusioni sull'attività svolta.
\\
\\
Per una maggiore comprensione dell'elaborato gli acronimi, le abbreviazioni e i termini ambigui o di uso non comune menzionati sono definiti nel Glossario, situato alla fine del documento.


%**************************************************************
\section{Convenzioni tipografiche}

Nei paragrafi di questa sezione sono riportate le norme tipografiche adottate durante la stesura del testo. La scelta di utilizzare delle norme specifiche ha lo scopo di produrre un documento formale e coerente.

\subsection{Stile del testo}
 Al fine di migliorare la leggibilità e comprensione di un documento, è d'obbligo preferire uno stile d'esposizione sfruttando elenchi piuttosto che uno stile narrativo, in maniera tale da esporre i contenuti più esplicitamente.
\begin{itemize}
	\item \textbf{Grassetto}: è utilizzato per:
	\begin{itemize}
		\item titoli;
		\item elementi di un elenco puntato che riassumono il contenuto del relativo paragrafo.
	\end{itemize}
	\item \textbf{Corsivo}: è utilizzato per:
	\begin{itemize}
		\item citazioni;
		\item abbreviazioni;
		\item nomi di aziende;
		\item termini stranieri da evidenziare.
	\end{itemize}
	\item \textbf{Maiuscolo}: le parole scritte interamente in maiuscolo si riferiscono soltanto ad acronimi o a nomi propri che lo richiedono.
	\newpage
	\item \textbf{Monospace}: le porzioni di testo scritte in monospace definiscono:
	\begin{itemize}
		\item frammenti di codice;
		\item comandi;
		\item \gls{url}.
	\end{itemize}
\end{itemize}

\subsection{Righe mal poste}
Le righe mal poste sono quelle che coincidono con una delle seguenti descrizioni:
\begin{itemize}
	\item una riga di un paragrafo (o un titolo di livello superiore o inferiore) che inizia alla fine di una pagina;
	\item una riga di un paragrafo (o un titolo di livello superiore o inferiore) che finisce all'inizio di una pagina.
\end{itemize}
Per una questione di leggibilità, queste tipologie di righe sono evitate.\\
 \subsection{Virgolette}
 Le virgolette sono utilizzate come segue:
 \begin{itemize}
 	\item \textbf{Virgolette singole ' '}: sono utilizzate solo per racchiudere un singolo carattere;
 	\item \textbf{Virgolette doppie " "}: sono utilizzate solo per racchiudere:
 	\begin{itemize}
 		\item citazioni;
 		\item nomi di documenti;
 		\item voci di un menù;
 		\item voci di pulsanti da premere.
 	\end{itemize}
 \end{itemize}

\subsection{Glossario}
La prima occorrenza dei termini riportati nel glossario è evidenziata in blu e contiene un riferimento al termine nel glossario, come nel seguente esempio: \gls{api};


\subsection{Elenchi puntati}
Gli elenchi puntati sono caratterizzati graficamente da un pallino nel primo livello, da un
trattino nel secondo e da un asterisco nel terzo.
Ogni elemento termina con il punto e virgola, tranne l'ultimo che termina con il punto. Ogni punto inizia con la lettera minuscola, tranne nel caso in cui necessiti una spiegazione.
\\
\\Esempio:
\begin{itemize}
	\item primo livello;
		\begin{itemize}
			\item secondo livello;
			\begin{itemize}
				\item terzo livello;
			\end{itemize}
		\end{itemize}
	\item Primo livello: primo livello dell'elenco.
\end{itemize}
% !TEX encoding = UTF-8
% !TEX TS-program = pdflatex
% !TEX root = ../tesi.tex

%**************************************************************
\chapter{Analisi retrospettiva}
\label{cap:analisi-retrospettiva}
%**************************************************************
Questo capitolo analizza l'attività di stage rispetto agli obiettivi raggiunti, ai benefici che le soluzioni ideate porteranno all'azienda e alle considerazioni finali dell'attività svolta.


%**************************************************************
\section{Obiettivi Raggiunti}

Gli obiettivi primari prefissati nel piano di lavoro sono stati completamente raggiunti. In seguito ad un'accurata analisi delle problematiche da risolvere è stato redatto un documento informale contenente a grandi linee le soluzioni che si intendevano sviluppare. Una volta approvate dal tutor aziendale, esse sono state sviluppate utilizzando dei Web Service come richiesto. Attraverso i moduli da aggiungere al \textit{Software Gestionale Vision} sono ora presenti dei nuovi metodi di attivazione e disattivazione delle licenze, seguiti dai controlli di validità. La creazione è invece gestita da \textit{License Manager 1.0}, rimpiazzando il Software \textit{GenPK} ed eliminando le problematiche ad esso correlate.
\\Anche gli obiettivi secondari e facoltativi sono stati raggiunti con successo. La data di scadenza può essere gestita da \textit{License Manager 1.0} e nella fase di avvio del \textit{Software Gestionale Vision} essa è controllata. Sempre tramite \textit{License Manager 1.0} è possibile gestire i moduli di una licenza, eliminando l'utilizzo del programma \textit{GenFileKey} e dei problemi ad esso correlati. Tramite la sezione "Log Accessi" del programma è possibile avere una panoramica completa dell'utilizzo del \textit{Software Gestionale Vision} comprese le anomalie, e nella sezione "Statistiche" è possibile osservare i dati riguardanti le licenze utili per scopi gestionali e di marketing.
\\Oltre agli obiettivi presentati nel piano di lavoro, durante lo stage ne sono stati fissati degli altri, grazie ad un veloce apprendimento delle tecnologie e a una buona progettazione. Attraverso \textit{License Manager 1.0} è stato deciso di poter gestire tutte le caratteristiche di una licenza, come un eventuale blocco, le componenti Hardware e l'indirizzo Email associato. Infine, poiché il programma è stato sviluppato utilizzando un sistema di utenti e presentava funzionalità spesso richieste anche dai rivenditori dell'azienda è stato deciso di modificare alcune sezioni per poter distribuire il Software ai rivenditori.
\\Il raggiungimento degli obiettivi formativi è illustrato nelle conclusioni.

%**************************************************************
\section{Benefici dello stage per l'azienda}
L'azienda, tramite l'attività di stage, può ora usufruire di un nuovo sistema di gestione e controllo delle proprie licenze, avendo i seguenti benefici:
\begin{itemize}
\item i problemi riguardanti le modalità di creazione delle licenze (inclusi i loro moduli) sono stati risolti;
\item è ora molto più difficile eludere la sicurezza come utilizzare licenze bloccate o la stessa licenza su più macchine grazie ai tre diversi controlli implementati (in modalità Online, Offline e durante l'esecuzione);
\item l'azienda, tramite \textit{License Manager 1.0} può ora gestire efficacemente con un solo Software le proprie licenze. Può infatti creare una licenza, gestirne i moduli, bloccarla, eliminarla, disattivarla e gestirne qualsiasi caratteristica tramite un'interfaccia semplice e intuitiva. Ha anche a disposizione un nuovo sistema di monitoraggio che permette di controllare possibili anomalie e avere informazioni utili per migliorare i propri prodotti;
\item i clienti possono gestire con maggiore autonomia la propria licenza, disattivandola e attivandola senza dover contattare l'azienda, anche in situazioni complicate come la rottura del proprio PC;
\item i rivenditori, attraverso \textit{License Manager 1.0}, possono gestire le licenze da loro vendute senza dover richiedere costantemente l'intervento di \textit{VISIONEIMPRESA s.r.l}. Le operazioni permesse ai rivenditori sono comunque ideate per preservare la sicurezza dell'azienda e ogni operazione è registrata e notificata all'azienda tramite email.
\end{itemize}

È infine importante sottolineare che avendo utilizzato dei Web Service e un Database per sviluppare le soluzioni previste è relativamente facile apportare modifiche all'intero sistema.

%**************************************************************
\section{Conclusioni}

L'esperienza di stage svolta è stata per me molto positiva. Entrare a contatto con programmatori esperti e con il mondo del lavoro è un'esperienza sicuramente molto formativa che non può essere insegnata in un contesto universitario. Anche la fase di progettazione e discussione con il tutor aziendale sulle soluzioni da me proposte sono state stimolanti, poiché analizzare i benefici e gli svantaggi di una decisione aiutano a migliorare le proprie capacità progettuali.
\\Il poter creare da zero un sistema di gestione e controllo mi ha aiutato a migliorare le mie capacità decisionali e di apprendimento, soprattutto perché nessun vincolo stretto era stato imposto. La creazione infatti, secondo me, è molto più stimolante della modifica, e per questo ho affrontato lo stage soddisfatto di quello che stavo creando, chiedendo aiuto ai programmatori più esperti in caso di difficoltà importanti e confrontandomi con il tutor aziendale per valutare la bontà delle soluzioni da me proposte.
\\A livello tecnologico ho acquisito una buona padronanza del linguaggio C\# e degli altri strumenti utilizzati nel corso dell'attività. Inoltre, entrando a contatto con un Software gestionale di buona qualità ho potuto acquisire conoscenze che mi saranno utili qualora volessi specializzarmi in questo settore.
\\L'unica nota negativa è da riferirsi alla posizione per me scomoda dell'azienda (circa quaranta minuti di auto da Padova centro), ma, poiché ho trovato il progetto molto interessante e grazie anche agli apprezzamenti ricevuti dal tutor aziendale alla fine dei lavori, sono particolarmente soddisfatto dell'attività di stage svolta.
% !TEX encoding = UTF-8
% !TEX TS-program = pdflatex
% !TEX root = ../tesi.tex

%**************************************************************
\chapter{Descrizione dello stage}
\label{cap:descrizione-stage}
%**************************************************************

Il capitolo riporta una descrizione dettagliata dell'attività di stage svolta, analizzando brevemente i possibili rischi, le problematiche da risolvere e le soluzioni proposte. Sono inoltre riportate la pianificazione del lavoro e i requisiti richiesti prima dell'inizio dello stage. 

%**************************************************************
\section{Analisi dei rischi}

Al fine di evitare rallentamenti dei periodi di lavoro è stata effettuata una breve analisi dei
rischi, in modo da evitare le situazioni che portano alla creazione di eventi non pianificati,
ove possibile. 
I rischi analizzati sono divisi per area di competenza, e per ognuno di essi è mostrata brevemente la strategia di mitigazione dello stesso.

\begin{itemize}
	\item livello tecnologico
	\begin{itemize}
		\item \textbf{Tecnologie sconosciute:} L'utilizzo di tecnologie sconosciute, come i servizi di Microsoft, potrebbero portare a ritardi o a difficoltà nello svolgimento del progetto. Tuttavia, grazie al periodo di apprendimento pianificato nel primo periodo dello stage e grazie alla presenza di programmatori esperti nel settore con cui confrontarsi, questo rischio non dovrebbe presentarsi;
		\item \textbf{Guasti hardware:} Durante lo svolgimento dello stage è possibile che la strumentazione utilizzata, in particolare il computer assegnato, possa incorrere in guasti hardware rischiando rallentamenti e perdita del lavoro. Per scongiurare questo evento, una copia delle soluzioni è salvata regolarmente in una cartella del server dell'azienda destinata allo studente.   
	\end{itemize}
	\item livello organizzativo
	\begin{itemize}
		\item \textbf{Valutazione delle risorse:} Data la poca esperienza con progetti di queste dimensioni si potrebbe incorrere in un'errata valutazione delle risorse, generando sprechi delle stesse o ritardi. Per mitigare questo rischio, ogni settimana avviene un incontro con il tutor aziendale per valutare il lavoro svolto e chiarire dubbi qualora si presentassero.
	\end{itemize}
	\item livello dei requisiti
	\begin{itemize}
		\item \textbf{Incomprensioni e scelte non ottimali:} è possibile che alcuni requisiti siano fraintesi o valutati erroneamente, portando allo sviluppo di un prodotto non consono alle aspettative dell'azienda. Per mitigare questo rischio, ogni settimana avviene un incontro con il tutor aziendale per valutare il lavoro svolto e chiarire dubbi qualora si presentassero.
	\end{itemize}
\end{itemize}

%**************************************************************
\section{Modalità di svolgimento}
L’attività di stage è stata svolta presso la sede dell’azienda per favorire l’interazione dello studente con il tutor e per affacciarlo nella realtà di un team di lavoro aziendale. Lo stagista ha avuto quindi la possibilità di confrontarsi con programmatori più esperti ed essere supportato al meglio in caso di problematiche di sviluppo e gestione del progetto.
Lo studente ha potuto confrontarsi con il tutor per qualsiasi problematica, mentre l’organizzazione settimanale del lavoro è stata gestita tramite dei meeting atti a definire lo stato di avanzamento del progetto e rivedere in tempo reale obiettivi settimanali o miglioramenti del prodotto sulla base dei risultati ottenuti dallo sviluppo.\\ I risultati sono stati valutati settimanalmente (o al termine dell’attività prevista) in base alla quantità e alla qualità dei prodotti forniti dallo studente.
L’orario lavorativo era il seguente: dal lunedì al venerdì dalle 8:40 alle 12:40 e dalle 13:30 alle 17:30.

%**************************************************************
\section{Cos'è una licenza Software}

Una licenza software è il contratto con il quale il titolare dei diritti sul software, di norma il produttore, concede all'utente il diritto di utilizzare il software, secondo i termini e le condizioni stabilite nel contratto stesso.
\\
Ogni installazione di software è accompagnata da una licenza. In un accordo di licenza può essere definito quanti utenti possono utilizzare il Software. Poichè ogni cliente necessita di esigenze diverse, esistono diverse tipologie di licenze del \textit{Software Gestionale Vision}, ognuna delle quali conferisce diverse funzionalità. 
\\
Le licenze software concedono all'utente il diritto d'uso sul prodotto software, sempre nel rispetto delle regole in esse contenute. I clienti in possesso di regolare licenza hanno la certezza di utilizzare il software originale, e rispettando le condizioni d'uso stabilite in fase di acquisto hanno la garanzia di essere conformi alle norme sul diritto d'autore e quindi di non incorrere in nessuna sanzione legale per violazioni della legge.
\\
L'acquisto di una licenza del \textit{Software Gestionale Vision} consiste nel ricevere un \texttt{Product Key}, ovvero una sequenza di 20 caratteri, univoca per ogni licenza, da utilizzare in fase di attivazione. Al \texttt{Product Key} è legato un \texttt{Serial Number}, composto da 6 cifre, e una \texttt{Tipologia}, che specificano quale licenza si ha acquistato e che funzionalità offre.
\\
Per semplicità, nel documento una licenza sarà trattata come l'insieme di \texttt{Product Key}, \texttt{Serial Number}, \texttt{Tipologia} e tutte le caratteristiche che la compongono, come la data di attivazione o il cliente a cui è stata venduta.

%**************************************************************
\newpage
\section{Problematiche da affrontare}
Nella presente sezione sono analizzate le problematiche da risolvere attraverso l'attività di stage. La descrizione delle problematiche è quindi da contestualizzarsi nel periodo antecedente lo stage.

\subsection{Creazione di una licenza}

La creazione di una nuova licenza avviene tramite il Software \textit{GenPK}. Esso offre la possibilità di creare uno o più \texttt{Product Key} a partire da \texttt{Serial Number} e \texttt{Tipologia} della licenza.
Il \texttt{Serial Number} può essere creato nei seguenti modi:
\begin{itemize}
\item \textbf{Manuale:} Sono utilizzate sei cifre scelte liberamente dal creatore. Se il numero scelto è già in uso è segnalato un errore;
\item \textbf{Casuale:} È fornito un numero a sei cifre casuale. Se il numero generato è già in uso è segnalato un errore; 
\item \textbf{Prime due cifre per l'anno:} Le prime due cifre corrispondono alle ultime due cifre dell'anno in corso, per tenere traccia dell'anno in cui il \texttt{Product Key} è stato generato. Al creatore è data la possibilità di scegliere il valore delle restanti quattro cifre, da cui partirà la ricerca per il primo \texttt{Serial Number} libero.
\end{itemize}
La \texttt{Tipologia} è scelta tra una delle seguenti:
\begin{itemize}
\item \textbf{LT}: gestionale semplice per le piccole aziende;
\item \textbf{ERP}: gestionale completo per le piccole e medie imprese;
\item \textbf{SQL:} estensione di ERP, con funzionalità ancora maggiori;
\item \textbf{Trasporti:} gestionale dedicato alle aziende di trasporto merci e persone, in particolare alla gestione dei carichi completi.
\end{itemize}

I product key creati sono salvati all'interno di un file, contenente la totalità dei dati delle licenze, con lo stato della licenza impostato a "da attivare". Il file delle licenze è scaricato all'avvio di \textit{GenPK} tramite server (GLOSSARIO)FTP aziendale ed è rinviato al server alla chiusura del programma. Lo scambio del file tramite protocollo FTP e l'avere il file a disposizione, seppur in modo riservato, sul proprio PC durante l'operazione espone l'intero sistema ad alti rischi come la corruzione del file con conseguente perdita di dati o la contraffazione degli stessi.\\
Infine, poiché i dati sono salvati su un unico file, per preservare la consistenza dei dati non è possibile avviare \textit{GenPK} su più macchine contemporaneamente.

\subsection{Attivazione di una licenza} 
Il cliente al primo avvio del \textit{Software Gestionale Vision} è invitato a inserire il \texttt{Product Key} ricevuto in fase d'acquisto. Il Software verifica che lo stato del \texttt{Product Key} sia impostato su "da attivare", preleva il (GLOSSARIO)MAC Address della scheda di rete su cui sta avvenendo l'installazione e, insieme al \texttt{Product Key}, attraverso un algoritmo di cifratura crea l’\texttt{Activation Key}, una stringa che sarà utilizzata per verificare che la licenza sia in uso sulla stessa macchina su cui è stata attivata. Lo stato del \texttt{Product Key}, dopo l'attivazione, viene modificato da “da attivare” ad “attivato”. Come nella creazione della licenza, i cambi di stato e la lista dei Product Key sono contenute in un file che il Software scarica sul PC in uso tramite protocollo FTP dal server dell’azienda, rinviato al server una volta modificato. Questa procedura, com'è già stato evidenziato nel paragrafo precedente, è molto rischiosa, poiché tutte le licenze sono disponibili su un file che potrebbe essere facilmente corrotto o manomesso.\\
Un esempio di attacco tramite questo sistema è poter usufruire di infinite licenze acquistandone due. 
Il meccanismo dell'attacco è semplice, ed è illustrato nei seguenti passaggi:
\begin{enumerate}
\item l'utente malevolo attiva il primo \texttt{Product Key} su una macchina. Lo stato del \texttt{Product Key} viene impostato correttamente da "da attivare" ad "attivato"; 
\item l'utente procede con l'attivazione del secondo \texttt{Product Key}, ma prima di terminare l'operazione il file contenente i dati delle licenze viene modificato, impostando lo stato del primo \texttt{Product Key} a "da attivare";
\item il primo \texttt{Product Key} è quindi disponibile per una nuova installazione.
\end{enumerate}
Questo tipo di attacco è permesso per due motivi. In primo luogo il file contenente i dati delle licenze è salvato sul PC dell'utilizzatore. Anche se cifrato non è difficile capire le stringhe corrispondenti ai termini "da attivare" e "attivato", ed è quindi possibile modificarlo senza conoscere il metodo di cifratura. In secondo luogo i controlli di validità della licenza, come il verificare che la licenza sia attiva su un solo PC per volta, sono svolti totalmente in locale. Quindi, potendo reinstallare il Software su una nuova macchina (ricreando quindi i dati per il controllo), i controlli saranno sempre superati, perché il sistema non ha modo di capire che la licenza è stata installata due volte, ma capirebbe solo se l'installazione del programma sarebbe copiata su un'altra macchina. 

\subsection{Avvio del Software Gestionale Vision} 

All'avvio del \textit{Software Gestionale Vision} è letta una chiave di registro contenente il \texttt{Product Key} della licenza, è prelevato il MAC Address della scheda di rete e insieme, con lo stesso algoritmo di cifratura utilizzato in fase di attivazione, è creato l'\texttt{Activation Key}. Il codice appena creato è controllato con l’Activation Key salvata in fase di attivazione in una chiave di registro; se è uguale allora la macchina utilizzata per accedere è la stessa su cui è stato installato il \textit{Software Gestionale Vision}, e l’utente può utilizzare il programma normalmente.\\
Poiché il controllo è svolto totalmente in locale esso è facilmente aggirabile attraverso l’uso di macchine virtuali identiche. Infatti, installando il \textit{Software Gestionale Vision} in una macchina virtuale e clonando quest'ultima, il controllo Hardware avrebbe sempre successo, potendo utilizzare di fatto una licenza su un numero illimitato di macchine. 
\\Un'altra problematica sorge da un qualsiasi guasto, o al cambio, della scheda di rete in quanto il controllo Hardware fallirebbe, e il cliente sarebbe costretto a contattare l’azienda per risolvere la situazione poiché l'\texttt{Activation Key} risultante sarebbe diversa, il che blocca l'esecuzione del Software.

\subsection{Caricamento dei moduli di una licenza} 
Il caricamento dei moduli di una licenza, ossia quali funzioni sono permesse, avviene da un file in formato “.HWK”, generato per mezzo del 
programma \textit{GenFileKey}, salvato nella cartella del Software Gestionale Vision. Esso contiene un codice cifrato, visibile all’utente, riepilogativo dei moduli attivi della licenza. Questo file può essere aperto con qualsiasi editor di testo e facilmente modificato, rischiando di compromettere la licenza.

\subsection{Licenze bloccate} 

Le licenze bloccate dall'azienda sono contenute in una Blacklist salvata all’interno del Software Gestionale Vision. Qualora l'azienda volesse aggiungere una nuova licenza nella Blacklist dovrebbe rilasciare un aggiornamento del proprio Software per far si che il programma riconosca la nuova licenza bloccata e non si avvii.
\\Questo provoca che un utente, non aggiornando il proprio programma (il che è consentito), può continuare a utilizzare il \textit{Software Gestionale Vision} anche in caso di licenza bloccata. 

\subsection{Procedura di disattivazione} 
Per disattivare la licenza dal proprio computer e reinstallarla in un altro il cliente è sempre costretto a contattare l'azienda. Questo provoca sia un onere non indifferente per l'azienda, soprattutto nei casi di rinnovamento delle macchine di un'azienda cliente, sia una condizione non ottimale da parte del cliente che si trova costretto a contattare l'azienda quando potrebbe acquisire autonomia con il click di un semplice pulsante.

\subsection{Scadenza di una licenza} 
Le licenze non possiedono una data di scadenza. L'acquisto avviene sottoscrivendo un contratto per l'utilizzo del Software e per disporre dell'assistenza tecnica, quindi un utente acquistando il Software una singola volta potrà sempre utilizzarlo. Solo in caso di nuovi aggiornamenti il cliente potrebbe voler riacquistare il Software, ma questo non è assolutamente necessario. 
\\Questa politica genera uno scarso controllo sullo stato delle licenze vendute, e inoltre limita i profitti dell'azienda. Vendere una licenza per un periodo definito piuttosto che per un tempo illimitato è una strategia molto utilizzata, accettata dai clienti e fornisce margini di guadagno molto migliori.

\subsection{Rivenditori} 
L'azienda \textit{VISIONEIMPRESA} oltre ai clienti finali gestisce dei rivenditori in grado di vendere il \textit{Software Gestionale Vision} a terzi. 
I rivenditori per operare, ad esempio per creare una licenza o gestirne i moduli, devono sempre contattare l'azienda, in modo che essa sia sempre aggiornata sulla situazione delle proprie licenze e che possa fatturare quanto venduto. I rivenditori quindi dispongono di un'autonomia molto limitata, generando situazioni poco confortevoli e oneri non necessari.

\subsection{Monitoraggio delle licenze}
L'azienda può monitorare lo stato delle licenze attraverso il Software \textit{GenPK}, ma le informazioni da esso mostrate sono minime, come il cliente associato e la tipologia, o il numero di licenze vendute per tipologia. Altre informazioni fondamentali, come i moduli di una licenza, non sono facilmente reperibili, e avere una visione completa, comprendente anche i rivenditori, aiuterebbe sicuramente ad ottenere una migliore gestione.



%**************************************************************
\section{Soluzioni proposte}

In seguito ai punti salienti esaminati in fase d’analisi, sono proposte (TODO per ora brevemente, da approfondire), per punti, le seguenti soluzioni:
\begin{itemize}

\item	Registrazione di una licenza: In primis si vuole eliminare il sistema dei Product Key salvati in un file, scaricato tramite ftp, costantemente a rischio. La generazione dei Product Key riprenderà il metodo di GenPK, ma i Product Key con le relative caratteristiche saranno salvati su un Database, e la Generazione avverrà tramite Web Service. L’Activation Key viene eliminata e si utilizzerà un nuovo sistema di controllo per l’impronta Hardware, basato su un insieme di componenti e non solo sulla scheda di Rete, in modo da permettere all’utente di cambiare alcune delle componenti del pc (ad esempio in caso di guasti) e non dover ricontattare l’azienda per poter continuare a utilizzare il Software Gestionale Vision. Per la creazione dei Product Key sarà creato un nuovo Software, chiamato License Manager 1.0, che permetterà di creare, gestire e monitorare le licenze in tutti i loro aspetti. Il Software utilizzerà Web Services e Database, eliminando la necessità di affidare file ai clienti con il rischio che siano compromessi. In fase di registrazione il cliente assocerà un indirizzo email alla propria licenza, in modo che possa disattivare e reinstallare il programma senza il bisogno di dover contattare l’azienda.
\item	Avvio del Software Gestionale Vision:  Sono stati pensati due diversi controlli, uno per un accesso senza connessione un per l’accesso con connessione. Il controllo per l’avvio del software in offline utilizzerà informazioni salvate in chiavi di registro per verificare la validità della licenza. E’ stato pensato anche un metodo basato sulle firme digitali per verificare l’integrità delle chiavi di registro. L’utilizzo del programma in offline sarà permesso per 15 giorni, dopo di ché verrà chiesto di connettersi a internet. Il controllo online avviene in due passaggi: uno all’avvio che controlla la validità della licenza in termini di data di scadenza, bloccaggio e componente Hardware, sempre tramite Web Service, e il secondo che si esegue a intervalli regolari di un’ora per verificare che l’utente non stia utilizzando il programma su macchine differenti ma con stesso Hardware (ad esempio clonando una macchina virtuale).
\item	Caricamento dei moduli della licenza: Il codice cifrato verrà salvato in un database, eliminando la necessità di inviare un file ai clienti, con possibilità che esso venga corrotto. La generazione del file che prima avveniva tramite GenFileKey ora sarà possibile eseguirla da License Manager 1.0.
\item	Licenze bloccate: salvando i dati della licenza su un database è facile segnalarne il blocco in uno dei suoi campi. Al momento dei controlli quel campo sarà analizzato, e in caso di blocco presente si deciderà che operazione intraprendere. 
\item	Procedura di disattivazione: associando un indirizzo email alla propria licenza l’utente sarà in grado di disattivare/reinstallare la propria licenza ogni volta che lo vorrà, senza contattare l’azienda e in qualsiasi situazione, anche in caso di rottura del pc o reinstallazione del sistema operativo.
\item	Scadenza di una licenza: sarà implementato un controllo sulla data di scadenza, sia in offline sia in online. Nel controllo all’avvio del programma sarà anche controllato se la licenza è entrata nell’ultimo mese di validità. In quel caso sarà mostrato un reminder con i giorni rimanenti prima della scadenza. 
\item	Rivenditori: License Manager 1.0, grazie al suo sistema di utenti, potrà essere distribuito ai rivenditori con utenti di tipo Guest, lasciando loro un certo grado di libertà che non li costringa a rivolgersi all’azienda per ogni decisione. Ogni azione da loro intrapresa sarà comunicata all’azienda tramite email, e tutti gli stati precedenti alle modifiche saranno registrati in una tabella dedicata del database.
\item	Monitoraggio delle licenze: License Manager 1.0 fornisce un sistema di monitoraggio basato su: visione immediata di tutte le caratteristiche di una licenza, statistiche sulla distribuzione e l’utilizzo delle licenze, log degli accessi al Software Gestionale Vision per notare possibili anomalie.


\end{itemize}

%**************************************************************
\section{Pianificazione del Lavoro}

La seguente sezione mostra la pianificazione del lavoro attuata prima di iniziare l'attività di stage e l'effettivo utilizzo delle ore. Per rendere più chiara la pianificazione del lavoro e lo svolgimento effettivo delle attività sono utilizzati Diagrammi di Gantt.

\subsection{Pianificazione antecedente lo stage}

TODO METTERE IN VERTICALE
\begin{figure}[!h] 
    \centering 
    \includegraphics[width=0.9\columnwidth]{ganttPrima} 
    \caption{Diagramma di Gantt - Pianificazione del lavoro}
\end{figure}

\subsection{Resoconto delle attività svolte}

TODO Diagramma post stage

%**************************************************************
\section{Obiettivi e Requisiti}

\subsection{Obiettivi}
Nella fase preliminare dello stage sono stati delineati i seguenti obiettivi, in ordine di importanza. Gli obiettivi sono identificati da un codice così composto: XXYY dove XX rappresenta la tipologia dell'obiettivo (es. OP obiettivo primario) e YY è un numero progressivo.
Le sigle sono queste.. TODO: sigle, espandere e spiegare nel dettaglio gli obiettivi.
\begin{itemize}

\item \textbf{obiettivi primari:}\begin{itemize}
\item OP01: definizione delle strategie risolutive per le problematiche presentate;
\item OP02: implementazione di un sistema di attivazione/disattivazione licenze via web services.
\end{itemize}
\item \textbf{obiettivi secondari:}\begin{itemize}
\item OS1: implementazione della data di scadenza di una licenza e relativi controlli (prototipo da completare).
\end{itemize}
\item \textbf{obiettivi facoltativi:}\begin{itemize}
\item OF01: implementazione licenze per moduli;
\item OF02: raccolta dati sull’attivazione delle licenze e loro utilizzo;
\item OF03: creazione dashboard con statistiche e alert su licenze in uso, disattivate, anomalie.
\end{itemize}
\item \textbf{obiettivi formativi:}\begin{itemize}
\item FO01: acquisizione di competenze utili allo sviluppo di software gestionale;
\item FO02: interazione con un team di lavoro aziendale;
\item FO03: ottenimento di capacità decisionali sulle migliori tecnologie da utilizzare in diversi contesti.
\end{itemize}

\end{itemize}

Durante lo svolgimento dello stage, dopo aver raggiunto tutti gli obiettivi prefissati, è stato posto come obiettivo ultimo la distribuzione del Software License Manager 1.0 anche ai rivenditori dell'azienda.

\subsection{Requisiti}

In relazione agli obiettivi presentati nel paragrafo precedente, sono stati identificati i requisiti riportati in seguito.

\subsubsection{Requisiti OP01}
Per raggiungere una completa (riassunto dell'obiettivo) sono stati identificati i seguenti requisiti:
descrizione dei requisiti da soddisfare per raggiungere l'obiettivo.
\subsubsection{Requisiti OP02} 
Per implementare un efficiente sistema di attivazione sono stati identificati i seguenti requisiti:
\begin{itemize}
\item l'utente finale deve poter inserire all'avvio del Software Gestionale Vision il Product Key, ricevuto in fase d'acquisto, relativo alla licenza per lui creata;
\item il Product Key inserito deve essere controllato tramite un metodo di un Web Service, che provvederà a verificare la disponibilità dello stesso e a impostare le informazioni di attivazione della licenza nella tabella \texttt{Licenze} del database \texttt{DBLicenze};
\item il modulo di attivazione, alla risposta del Web Method, deve impostare le chiavi di registro necessarie per il controllo della licenza in modalità Offline;
\item ecc...
\end{itemize} 
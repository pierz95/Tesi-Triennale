% !TEX encoding = UTF-8
% !TEX TS-program = pdflatex
% !TEX root = ../tesi.tex

%**************************************************************
\chapter{Moduli Software Gestionale Vision}
\label{cap:moduli-vision}
%**************************************************************
In questo capitolo sono illustrati nel dettaglio i moduli da aggiungere al \textit{Software Gestionale Vision} per implementare correttamente il nuovo sistema di gestione e controllo delle licenze Software. I moduli, come \textit{License Manager 1.0}, sfruttano i Web Service e il Database illustrati nel capitolo {\hyperref[cap:sviluppo-software]{Web Service e Database}} per garantire sicurezza e affidabilità. 


\section{Premesse}
I moduli saranno implementati dai programmatori modificando gli algoritmi o le chiavi di cifratura scelti per preservare la sicurezza dell'azienda. Tuttavia alcuni funzionamenti, come il controllo delle componenti Hardware associate a una licenza, rimarranno invariati, e pertanto non potranno essere illustrati nel dettaglio in questo documento.
\\Poiché le funzionalità principali sono svolte dai Web Service i moduli, sviluppati in C\#, potranno essere facilmente migrati in altri linguaggi di programmazione utilizzati dall'azienda.
\\
Il \textit{Software Gestionale Vision} utilizza delle chiavi di registro per salvare le informazioni di una licenza, come il \texttt{Product Key}. Per i moduli sviluppati si è deciso di utilizzare la stessa tecnica. Esse saranno specificate nel corso della trattazione. 
\\La verifica delle componenti Hardware associate a una licenza per verificarne l'utilizzo su un unico computer è approfondita nella sezione {\hyperref[cap:appA]{Grado di compatibilità}}.  
\\Nei prossimi paragrafi sono illustrati tutti i moduli sviluppati.

%**************************************************************
\newpage
\section{Attivazione di una licenza}
Il modulo di attivazione creato non sarà aggiunto, bensì rimpiazzerà quello attualmente in uso, poiché il \textit{Software Gestionale Vision} permette già di attivare una licenza.\\
La nuova procedura di attivazione di una licenza utilizza un Web Service invece del file scambiato tramite connessione FTP, aumentando la sicurezza e diminuendo il rischio di perdita dei dati. Questo implica che al momento della registrazione il computer debba essere connesso a internet.\\
Dopo che il cliente ha acquistato la licenza e ha ricevuto il \texttt{Product Key} dall'azienda, l'attivazione avviene tramite i seguenti passaggi:
\begin{enumerate}
\item il programma chiede di inserire il \texttt{Product Key} ricevuto;
\item il programma preleva le componenti Hardware utilizzate per controllare che la licenza sia in utilizzo su un unico PC, e insieme al \texttt{Product Key} invoca un metodo del Web Service \texttt{LicenseManagerService} per controllare se il \texttt{Product Key} inserito esista e possa essere attivato;
\item il Web Service, dopo il controllo, risponde con un codice che ne specifica l'esito, e in base ad esso il programma procede diversamente:
\begin{itemize}
\item ad un codice di errore il programma avvisa l'utente e la procedura termina. Un errore incorre nel caso di: \begin{itemize}
\item errore del Web Service;
\item il \texttt{Product Key} inserito non esiste;
\item il \texttt{Product Key} inserito è associato a una licenza bloccata;
\item il \texttt{Product Key} inserito è associato a una licenza scaduta.
\end{itemize}
\item ad un codice di successo, il programma procede con l'attivazione della licenza in base al codice ricevuto. I codici di successo sono i seguenti:
\begin{itemize}
\item il \texttt{Product Key} esiste, la licenza associata è valida e non ha nessuna componente Hardware collegata. La continuazione dell'attivazione è illustrata nella sezione \ref{nha};
\item il \texttt{Product Key} esiste, la licenza associata è valida e la componente Hardware collegata è compatibile con le componenti inviate per il controllo. La continuazione dell'attivazione è illustrata nella sezione \ref{hc};
\item il \texttt{Product Key} esiste, la licenza associata è valida e la componente Hardware collegata non è compatibile con le componenti inviate per il controllo. La continuazione dell'attivazione è illustrata nella sezione \ref{hnc}.
\end{itemize}
\item un codice speciale è fornito qualora solo una specifica componente Hardware inviata per il controllo è uguale a quelle utilizzate. Questo codice rappresenta una situazione anomala e i dettagli del funzionamento sono riportati nella sezione \ref{off} relativa al controllo in modalità Offline. Le azioni da intraprendere alla ricezione di questo codice saranno decise dall'azienda al momento dell'implementazione, attualmente è mostrato un messaggio d'errore e la procedura d'attivazione è bloccata.
\end{itemize}
\end{enumerate}

Illustriamo ora le diverse modalità di attivazione in base al codice di successo ricevuto.


\subsection{Nessun Hardware associato}
\label{nha}

Nel caso non risulti nessun Hardware associato alla licenza si controlla che non sia presente nessun’altra licenza associata alla macchina. Il controllo è eseguito verificando che non esista la chiave di registro relativa al \texttt{Product Key} (creata durante un'attivazione). Se è già presente un’altra licenza si chiede di disinstallarla prima di poter continuare.
\\La procedura prosegue chiedendo all’utente di associare un indirizzo Email necessario per la disattivazione e riattivazione della licenza in autonomia. 
Per essere sicuri che l’utente stia registrando un indirizzo email valido e a lui accessibile è inviato un codice casuale a quell’indirizzo, e l’utente deve inserirlo nella casella apposita per confermarne la ricezione. 
\\Una volta confermata la validità dell'indirizzo email la procedura prosegue invocando il metodo del Web Service \texttt{LicenseManagerService} responsabile di svolgere tutte le operazioni relative a una prima attivazione, come impostare la data di attivazione, le componenti Hardware e l'indirizzo Email associato. Se il metodo è stato eseguito con successo è utilizzato il Web Service \texttt{LicenseSecurityService} per ottenere le caratteristiche della licenza da salvare in chiavi di registro e da utilizzare per il controllo in modalità Offline spiegato nella sezione \ref{off}. 


\subsection{Hardware compatibile}
\label{hc}
Nel caso le componenti Hardware controllate risultino compatibili con quelle attualmente associate alla licenza è molto probabile che l'utente abbia reinstallato Windows, rimuovendo il \textit{Software Gestionale Vision}, o eliminato le informazioni che permettono al Software di rilevare l'avvenuta attivazione della licenza. 
\\La procedura permette quindi di proseguire controllando che l'utente sia lo stesso che ha eseguito la prima attivazione. Il controllo è lo stesso utilizzato per associare un indirizzo Email alla licenza, ossia è generato un codice casuale ed è inviato all'indirizzo Email associato alla licenza. L'utente è quindi costretto a inserire il codice ricevuto per continuare l'attivazione.
\\Se il controllo di autenticità dell'utente è superato, è utilizzato il metodo del Web Service \texttt{LicenseManagerService} responsabile di aggiornare le informazioni e le date derivanti dalla nuova attivazione. In caso di successo del metodo, è invocato il metodo del Web Service \texttt{LicenseManagerService} per resettare il valore dell’\texttt{UniqueID} utilizzato per il controllo in modalità Online durante l’esecuzione del \textit{Software Gestionale Vision} (vedere la sezione \ref{uniqueid} per maggiori informazioni).\\
L’esecuzione termina utilizzando il Web Service \texttt{LicenseSecurityService} per ottenere le caratteristiche della licenza da salvare in chiavi di registro e da utilizzare per il controllo in modalità Offline spiegato nella sezione \ref{off}. 


\subsection{Hardware non compatibile}
\label{hnc}
Nel caso le componenti Hardware controllate risultino incompatibili con quelle attualmente associate alla licenza è probabile che l'utente voglia reinstallare la propria licenza su un computer diverso del precedente, ad esempio per un guasto improvviso o l'impossibilità di usare di quest'ultimo, non potendo disinstallare la licenza da esso. 
\\La procedura controlla che non sia presente nessun’altra licenza associata alla macchina. Il controllo è eseguito verificando che non esista la chiave di registro relativa al \texttt{Product Key} (creata durante un'attivazione). Se è già presente un’altra licenza si chiede di disinstallarla prima di poter continuare, altrimenti è segnalato all’utente che la licenza che si sta cercando di installare è già installata su un altro computer, e viene chiesto se la si vuole dissociare da quella macchina per reinstallarla su quella corrente. A una risposta positiva dell’utente il programma procede con l’attivazione controllando che l’utente che sta tentando la disattivazione per riattivare la licenza sia effettivamente il possessore dell'indirizzo Email associato alla licenza. Il controllo è lo stesso utilizzato dalle altre due tipologie di attivazione, ossia è generato un codice casuale ed è inviato all'indirizzo Email associato alla licenza. L'utente è quindi costretto a inserire il codice ricevuto per continuare l'attivazione.
\\
Se il controllo di autenticità dell'utente è superato, è utilizzato il metodo del Web Service \texttt{LicenseManagerService} responsabile di aggiornare le date e le informazioni derivanti dalla nuova attivazione, come ad esempio le nuove componenti Hardware. In caso di successo del metodo, è invocato il metodo del Web Service \texttt{LicenseManagerService} per resettare il valore dell’\texttt{UniqueID} utilizzato per il controllo in modalità Online durante l’esecuzione del \textit{Software Gestionale Vision} (vedere la sezione \ref{uniqueid} per maggiori informazioni).\\
L’esecuzione termina utilizzando il Web Service \texttt{LicenseSecurityService} per ottenere le caratteristiche della licenza da salvare in chiavi di registro e da utilizzare per il controllo in modalità Offline spiegato nella sezione \ref{off}. 


\section{Avvio del Software e controlli}

All’avvio del \textit{Software Gestionale Vision} deve essere verificata la validità della licenza. Il controllo e le operazioni da eseguire sono differenti a seconda se è disponibile la connessione a internet. 

%**************************************************************
\subsection{Avvio del software in modalità Online}

Il controllo di validità della licenza all'avvio, se il computer è connesso a Internet, avviene in due passaggi, sfruttando i dati del Database e non quelli salvati in locale, in quanto possono essere facilmente corrotti. Il primo controllo riguarda la validità della licenza in sé, il secondo è attuato per controllare che la licenza non sia in uso su un'altra macchina identica (ad esempio clonando una macchina virtuale).\\
Il controllo inizia con il prelevare la chiave di registro contenente il \texttt{Product Key}. Se non è presente il controllo non continua, poiché non sembra essere attivata nessuna licenza su quel computer.\\  
Il controllo di validità avviene tramite il Web Service \texttt{LicenseManagerService}, che ritorna come risultato di successo un array di byte contenenti le caratteristiche della licenza da utilizzare per il controllo in modalità Offline. Qualora la licenza non fosse valida, viene mostrato il motivo dell’errore e il controllo fallisce.
\\Le caratteristiche, in caso di successo, sono cifrate e salvate in chiavi di registro e sono le seguenti: 
\begin{itemize}
\item Hash della stringa formata dalle caratteristiche, firmato dal Server con la chiave privata;
\item Product Key;
\item Data di scadenza;
\item Data di ultimo accesso;
\item Data di ultimo controllo Online;
\item Data di fine del periodo permesso in Offline;
\item Componente Hardware specifica;
\item Chiave pubblica del Server;
\end{itemize}

La componente Hardware specifica è una componente che per motivi di sicurezza non è qui specificata.\\
Attualmente è utilizzato \gls{AES} come algoritmo di cifratura, ma l'azienda \textit{VISIONEIMPRESA s.r.l.} ne utilizzerà un altro al momento dell'implementazione nel \textit{Software Gestionale Vision}.

Il controllo di unicità della macchina utilizzata per la licenza avviene creando un \gls{thread} che esegue ogni ora. Il thread in questione esegue la procedura illustrata nella sezione \ref{uniqueid}.\\

Il modulo come operazioni finali dell’avvio in modalità Online: 
\begin{itemize}

\item controlla che la licenza non stia per scadere, e in caso avvisa l’utente comunicando i giorni rimanenti;
\item aggiunge il nuovo accesso al \textit{Software Gestionale Vision} utilizzando il Web Service \texttt{LicenseManagerService}, prelevando l'\texttt{indirizzo IP} e il \texttt{MAC Address} della scheda di rete;
\item imposta la data di ultimo accesso al \textit{Software Gestionale Vision} attraverso il Web Service LicenseManagerService.

\end{itemize}

%**************************************************************
\subsection{Avvio del software in modalità Offline}
\label{off}

All'avvio di Vision deve essere possibile controllare la validità della licenza anche senza connettività, altrimenti, lavorando perennemente Offline, si potrebbe eludere il sistema sia in termine di date sia nell'univocità del pc di utilizzo.\\
Due procedure di controllo sono svolte all'avvio del programma in modalità Offline: una di controllo della data di scadenza, l'altra di controllo della componente Hardware. Ogni volta che un qualsiasi controllo fallisce, il programma richiede all'utente di connettersi a Internet per risincronizzare i dati e tentare di risolvere la situazione. È dato un periodo di 15 giorni in cui il programma può essere usato in Offline, al termine del quale bisogna connettersi per verificare la validità della licenza.\\
Il controllo inizia prelevando le chiavi di registro contenenti le informazioni salvate durante il controllo in modalità Online. Se una delle chiavi dovesse essere stata rimossa è mostrato un errore e la procedura termina. Dopo aver estratto i valori delle chiavi, esse vengono decifrate e si procede con la verifica dell’integrità delle chiavi. Per verificare che l’utente non abbia modificato appositamente le chiavi (sebbene sia difficile poiché cifrate) è presente una chiave di registro contenente l’Hash delle caratteristiche inviate, firmato con la chiave privata del server.
La verifica dell'integrità si basa sulla crittografia a chiave pubblica. In breve, esiste una coppia di chiavi (salvata in un \gls{KeyContainer} del Server) formata da una chiave pubblica e una privata. La chiave pubblica può essere distribuita liberamente, mentre la chiave privata deve essere custodita segretamente. Un messaggio cifrato con una chiave può essere decifrato con l'altra, e non è possibile risalire alla chiave privata partendo dalla chiave pubblica. L'integrità sfrutta questa tecnologia nel modo seguente:
\begin{enumerate}
\item l'insieme delle caratteristiche, disposte in un ordine specifico, viene trasformato in Hash, e in seguito è cifrato attraverso la chiave privata (viene firmato);
\item le caratteristiche e l'Hash sono inviate al Client che esegue il controllo;
\item il Client ricrea l'Hash utilizzando lo stesso ordine specifico e controlla che sia uguale all'Hash ottenuto decifrando l'Hash firmato, utilizzando la chiave pubblica del Server. Questa verifica può essere fatta poiché nessuno è in grado di creare un messaggio decifrabile dalla chiave pubblica (che è quindi cifrato con la chiave privata tenuta segretamente dal Server) al di fuori del Server, e quindi l'Hash è da ritenersi valido e creato dal server. Se il Client non è in grado di decifrare l'Hash firmato o ricreare l'Hash uguale a quello inviato dal Server allora il controllo fallisce.
\end{enumerate}
L'algoritmo di cifratura a chiave pubblica utilizzato è \gls{RSA}.

Superato il controllo dell'integrità delle chiavi il modulo continua con il controllo della data di scadenza e della componente Hardware.

\subsubsection{Controllo della data di scadenza}

Per controllare la data di scadenza sono utilizzate le seguenti quattro chiavi, ognuna di esse cifrata, che devono essere decifrate prima di procedere col controllo. 
\begin{itemize}
\item \textbf{Data dell'ultimo controllo:} coincide inizialmente con la data di attivazione, e si aggiorna ad ogni accesso ad Internet indicando la data di ultimo accesso al \textit{Software Gestionale Vision} in modalità Online;
\item \textbf{Data di fine periodo Offline:} è impostata come data dell'ultimo controllo aumentata di 15 giorni. È ricalcolata ad ogni accesso del \textit{Software Gestionale Vision} in modalità Online;
\item \textbf{Data di scadenza:} è la data di scadenza della licenza;
\item \textbf{Data di ultimo accesso:} è la data utilizzata per verificare che l'utente non stia portando indietro l'orologio del pc. Inizialmente è impostata uguale alla data dell'ultimo controllo, e viene aggiornata ad ogni avvio in modalità Offline del programma se si soddisfa la seguente condizione: quando un utente accede al programma si controlla che la data attuale sia  maggiore della data di ultimo accesso. Se la data è minore è molto probabile che l'utente abbia portato indietro la data del pc. Il programma comunica quindi un'irregolarità rilevata, pregando di ricontrollare la data o di contattare l'azienda.
\end{itemize}
Qualora una delle seguenti chiavi mancasse o fosse stata corrotta, è possibile accedere al \textit{Software Gestionale Vision} in modalità Online in modo da aggiornare le chiavi di registro con i dati corretti.
Il programma controlla quindi che la data attuale sia minore della data di scadenza, e in caso negativo non permette l’accesso. Infine viene controllato anche che la data attuale non sia maggiore della data di fine periodo Offline, il che significherebbe che l’utente ha terminato i 15 giorni di utilizzo a disposizione.

\subsubsection{Controllo della componente Hardware}

Il controllo della compatibilità Hardware è realizzato in modo diverso da quello utilizzato dal Web Service \texttt{LicenseSecurityService}, per conservarne la riservatezza. In questo controllo è semplicemente verificata un'unica specifica Hardware estraendola dalla chiave di registro corrispondente, e il programma permette l'accesso se la componente controllata è la stessa del computer in uso. Controllare solo una specifica Hardware da un lato dà un certo grado di affidabilità, in quanto la componente scelta è molto difficile che sia sostituita, dall'altro essendo un controllo diverso permette un monitoraggio sul fallimento di un controllo lato Online ma che sarebbe superato in modalità Offline, il che produce un segnale d'allarme. Questa considerazione si basa sul fatto che è poco probabile che un utente cambi tutto il PC mantenendo solo la componente scelta. È più probabile che abbia capito che nel controllo Offline è controllata quella componente e abbia tentato di forzare il sistema, non sapendo che nel controllo in modalità Online le componenti Hardware considerate sono molte altre.


%**************************************************************
\section{Controllo della licenza durante l'esecuzione}
\label{uniqueid}

E' possibile aggirare il controllo Hardware installando il software in una macchina virtuale e clonando quest'ultima. Per risolvere questo problema si propone lo sviluppo di un semplice servizio web per la verifica. Questo comporta che tale controllo possa avvenire solo quando il PC è connesso a Internet.\\
Il funzionamento del servizio è relativamente semplice. Si tiene traccia di un ID univoco (\texttt{UniqueID}), conosciuto dal Server e dal Client. Ad intervalli regolari (nel caso in esame un’ora ma può essere diminuito per aumentare la sicurezza), il programma interroga il Server mandando l'\texttt{UniqueID} salvato sul PC. Se questo ID trova corrispondenza nel Server allora l'utilizzatore è verificato, e il Server risponde inviando un altro \texttt{UniqueID} che sarà utilizzato per il controllo successivo, e cosi via.
\\Supponiamo che l'utente provi a clonare la licenza installandola su una macchina virtuale. Entrambe le macchine, poiché identiche, supererebbero il controllo Hardware. Analizziamo ora il controllo che vogliamo implementare. La prima macchina, connettendosi al server per lo scambio degli ID, otterrebbe una risposta positiva, provocando il cambiamento dell'ID contenuto nel Server. Se la seconda macchina provasse a connettersi al server con l'ID in essa salvato (che coincide con quello che ha utilizzato la prima macchina poco prima) il controllo fallirebbe, e il programma si bloccherebbe. E' possibile che la sincronizzazione venga persa ad esempio per l'eliminazione della chiave di registro dovuta a vari problemi. A questo scopo è inserito un contatore nel database che conta quante volte la chiave contente l'ID non viene trovata. Se si perde per più di tre volte in un mese, il programma procede al bloccaggio della licenza.\\
Questo controllo in caso di licenze disponibili per più utenti deve essere implementato sulla macchina Server che fornisce servizio alle altre o essere omesso, altrimenti sarebbe impossibile il corretto funzionamento del sistema.\\
Il procedimento appena illustrato è realizzato dapprima prelevando il \texttt{Product Key} dalla chiave di registro corrispondente, e solo se è presente si continua con il controllo. Il programma estrae l'\texttt{UniqueID} dalla chiave corrispondente. Se la chiave non è presente è invocato un metodo del Web Service \texttt{LicenseSecurityService} per risincronizzare l’\texttt{UniqueID}, fino ad un massimo di tre volte. Se la chiave di registro è presente il suo valore viene decifrato e si controlla che sia uguale a quello salvato nel Server attraverso il metodo dedicato del Web Service \texttt{LicenseSecurityService}. In caso di successo il metodo ritorna un nuovo \texttt{UniqueID} che viene salvato nella chiave di registro apposita, altrimenti viene ritornato un messaggio d’errore e il comportamento da adoperare sarà deciso dai programmatori in fase di implementazione. Attualmente il programma notifica il fallimento del controllo inviando una mail all'azienda tramite l'utilizzo del Web Service \texttt{LicenseEmailService}. 


%**************************************************************
\section{Disattivazione della licenza}

Con questo modulo viene data la possibilità di dissociare la propria licenza dal computer in uso, e quindi poter riutilizzare il \texttt{Product Key} ricevuto all’acquisto per reinstallare il \textit{Software Gestionale Vision}.\\
La disattivazione inizia controllando che la chiave di registro contenente il \texttt{Product Key} della licenza esista. Se non viene trovata la procedura si interrompe.\\
La disattivazione prosegue controllando che l'utente sia lo stesso che ha eseguito l'attivazione. Il controllo è lo stesso utilizzato per l'attivazione, ossia è generato un codice casuale ed è inviato all'indirizzo Email associato alla licenza. L'utente è quindi costretto a inserire il codice ricevuto per continuare la disattivazione.
\\Se il controllo di autenticità dell'utente è superato, l’esecuzione procede invocando il metodo responsabile della disattivazione del Web Service \texttt{LicenseManagerService}. Se la dissociazione  avviene con successo, il \texttt{Product Key} con cui riattivare la licenza è inviato tramite mail all’indirizzo email associato alla licenza.\\
Il modulo conclude l'esecuzione eliminando le chiavi di registro utilizzate per il controllo in modalità Offline e forzando il reset dell’\texttt{UniqueID}, per una prossima installazione, utilizzando il Web Service \texttt{LicenseSecurityService}. 



%**************************************************************
\section{Modifica dell'indirizzo Email della licenza}

Con questo modulo viene data all'utente la possibilità di cambiare l'indirizzo Email associato alla licenza.\\
Il cambio dell’email inizia controllando che la chiave di registro contenente il \texttt{Product Key} della licenza esista. Se non viene trovata la procedura si interrompe. Successivamente richiede all'utente di inserire il nuovo indirizzo Email. La procedura prosegue controllando che l'utente sia lo stesso che ha eseguito l'attivazione. Il controllo è lo stesso utilizzato per l'attivazione, ossia è generato un codice casuale ed è inviato all'indirizzo Email associato alla licenza. L'utente è quindi costretto a inserire il codice ricevuto per continuare la procedura.\\
Se il codice casuale è verificato viene invocato il metodo del Web Service\\ \texttt{LicenseManagerService} incaricato di impostare il nuovo indirizzo Email associato alla licenza.

%**************************************************************
\section{Analisi critica del sistema di controllo}

Il nuovo sistema di controllo sviluppato porta con sé molte migliorie rispetto a quello fino ad allora in vigore, pressoché inesistente. Le licenze infatti non avendo una data di scadenza potevano essere utilizzate senza limiti, gravando sui guadagni dell'azienda e riducendo la possibilità di gestione. Il controllo della componente Hardware è stato modificato per permettere all'utente di poter apportare modifiche non radicali al proprio PC e poter continuare ad utilizzare il \textit{Software Gestionale Vision}. Il vecchio sistema infatti, basandosi solo sulla scheda di rete, richiedeva molta assistenza da parte dell'azienda nelle procedure di disattivazione, poiché non è cosi raro un suo guasto o un cambiamento per motivi aziendali.
\\È stato implementato un controllo all'avvio del \textit{Software Gestionale Vision} in modalità Online, il quale verifica che la licenza non sia bloccata (eliminando il sistema poco pratico della \textit{blacklist}), che non sia scaduta, e che le componenti Hardware siano compatibili. Nessuno di questi controlli era eseguito dal vecchio sistema, tranne la verifica Hardware, ma avvenendo sempre in locale poteva essere aggirata nel seguente modo:
\begin{enumerate}
\item si installa la licenza sulla prima macchina;
\item dopo un po' di tempo (per non destare sospetti) si chiede all'azienda la disattivazione della stessa, il che non comporta l'eliminazione dell'\texttt{Activation Key} dato che è salvata solo in locale, bensì viene solo impostato il \texttt{Product Key} come disponibile per un'attivazione;
\item l'azienda ora non ha più modo di verificare che il \textit{Software Gestionale Vision} non sia in uso sulla prima macchina, ed è costretta a fidarsi della correttezza dei clienti. La licenza adesso può essere reinstallata in un nuovo computer, e rieseguendo la procedura si può disporre dell'utilizzo della licenza in più PC.
\end{enumerate}
Il nuovo sistema elimina questo problema se il Software è eseguito in modalità Online, ma il problema può persistere in Offline. Questo problema è dato poiché è pressoché impossibile riconoscere l'unicità di una macchina utilizzando solo dati contenuti in essa: che siano salvati in chiavi di registro, file nascosti o cifrati, è sufficiente eseguire una clonazione tramite l'utilizzo di macchine virtuali, e le due macchine risulteranno essere identiche se viste dall'interno.
\\
La clonazione delle macchine virtuali è controllata attraverso il sistema dell'\texttt{UniqueID}, in modo da segnalare immediatamente l'azienda al verificarsi di una anomalia. Questo è possibile se si sta utilizzando il Software in modalità Online, ma non in Offline.
\\
Il problema dell'unicità del computer in cui è eseguito il Software potrebbe essere risolto completamente, o per lo meno diventerebbe molto più sicuro, se il \textit{Software Gestionale Vision} richiedesse la presenza della connessione Internet per essere eseguito, ma per l'azienda è importante che i clienti possano utilizzare il loro Software anche in un periodo di vacanza o in caso di errori di connessione. Permettendo un periodo di 15 giorni in modalità Offline si limita leggermente il rischio di contraffazione, ma salvando lo stato della macchina virtuale all'inizio del periodo Offline e ricaricandolo una volta scaduto i controlli saranno superati, poiché in modalità Offline essi si basano su informazioni contenute in locale che sfruttano lo stato della macchina, e riportando il sistema a uno stato in cui i controlli erano superati essi saranno sempre superati, e finché il computer non viene connesso a Internet non è possibile attuare nessun tipo di controllo.
\\Concludendo, concedere la possibilità di lavorare in Offline è il principale rischio di aggiramento dei controlli. Tuttavia l'utilizzo e la clonazione di macchine virtuali non sono pratiche comuni e di facile comprensione per tutti i clienti dell'azienda; il sistema deve essere ben conosciuto prima di poter essere aggirato, e al primo avvio in modalità Online l'azienda sarebbe subito notificata delle possibili anomalie. Pertanto, sebbene il sistema non sia inespugnabile (premettendo che sarebbe comunque impossibile crearne uno, poiché con tecniche di \gls{Reverse Engineering} si potrebbe ricompilare il programma eliminando i controlli), esso fornisce sicuramente un migliore controllo rispetto al sistema precedentemente in uso. Inoltre, esso lascia soddisfatti la maggior parte dei clienti, presumibilmente onesti, che richiedono di poter lavorare anche in caso di necessità senza connessione a Internet. Anche il controllo in modalità Online potrebbe essere superato modificando i dati inviati o ricevuti dal Server, ma utilizzando una connessione HTTPS per l'utilizzo dei Web Service questo pericolo può non essere tenuto in considerazione.
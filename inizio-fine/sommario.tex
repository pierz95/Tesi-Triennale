% !TEX encoding = UTF-8
% !TEX TS-program = pdflatex
% !TEX root = ../tesi.tex

%**************************************************************
% Sommario
%**************************************************************
\cleardoublepage
\phantomsection
\pdfbookmark{Sommario}{Sommario}
\begingroup
\let\clearpage\relax
\let\cleardoublepage\relax
\let\cleardoublepage\relax

\chapter*{Sommario}

Il presente documento descrive il lavoro svolto durante il periodo di stage, della durata di 320 ore, dal laureando Pier Paolo Tricomi presso l'azienda \textit{VISIONEIMPRESA s.r.l.} di Pernumia (PD).
\\
L'attività di stage presentava diversi obiettivi. In primo luogo l'azienda richiedeva un'analisi dell'attuale sistema di creazione delle licenze del Software Gestionale Vision da loro venduto, per approfondirne il funzionamento e le debolezze. Successivamente, l'azienda richiedeva l'implementazione di un nuovo sistema, sviluppato tramite Web Service, e un'applicazione Desktop, in grado di creare nuove licenze e salvarle all'interno di un Database. Infine era richiesto lo sviluppo di alcuni moduli, sempre tramite l'utilizzo di Web Service, da aggiungere in un secondo momento al Software Gestionale Vision per il controllo della validità di una licenza, in termini di scadenza o di possibili contraffazioni, e per fornire all'utente finale maggiore libertà di gestione.\\
Per le souzioni da proporre, il candidato avrebbe potuto riferirsi a soluzioni e prototipi già sviluppati dai programmatori dell'impresa in contesti simili.
\\Nel corso dello stage gli obiettivi primari sono stati raggiunti in tempi minori di quelli previsti, il che ha portato ad ampliare il Software pensato per la creazione con ulteriori funzionalità di gestione e monitoraggio, anche per i rivenditori dell'azienda fino ad allora esclusi.
%\vfill
%
%\selectlanguage{english}
%\pdfbookmark{Abstract}{Abstract}
%\chapter*{Abstract}
%
%\selectlanguage{italian}

\endgroup			

\vfill



\label{cap:glossario}
%**************************************************************
% Acronimi
%**************************************************************
%\renewcommand{\acronymname}{Acronimi e abbreviazioni}

%\newacronym[description={\glslink{urlg}{Uniform Resource Locator}}]
  %  {url}{URL}{Uniform Resource Locator}    
    
%\newacronym[description={\glslink{ideg}{Integrated development environment}}]
%	{ide}{IDE}{Integrated development environment}
%
%\newacronym[description={\glslink{dbmsg}{Database Management System}}]{dbms}{DBMS}{Database Management System}

%**************************************************************
% Glossario
%**************************************************************
%\renewcommand{\glossaryname}{Glossario}

\newglossaryentry{dbms}
{
	name={DBMS},
	text= DBMS,
	sort=dbms,
	description= {Gestione di basi di dati, è un sistema Software in grado di gestire collezioni di dati che siano grandi, condivise e persistenti, assicurando la loro affidabilità e riservatezza}
}

\newglossaryentry{URL}
{
	name={URL},
    text=URL,
    sort=url,
    description={Sequenza di caratteri che identifica univocamente l'indirizzo di una risorsa in Internet, come un documento o un'immagine}
}

\newglossaryentry{ide}
{
	name={IDE},
	text=IDE,
	sort=ide,
	description ={In informatica un ambiente di sviluppo integrato (ing. \textit{integrated development environment}) è un software che, in fase di programmazione, aiuta i programmatori nello sviluppo del codice sorgente di un programma. Spesso l'IDE aiuta lo sviluppatore segnalando errori di sintassi del codice direttamente in fase di scrittura, oltre a tutta una serie di strumenti e funzionalità di supporto alla fase di sviluppo e debugging}
}

\newglossaryentry{ftp}
{
	name = FTP,
	text = FTP,
	sort = ftp,
	description = {\textit{File Transfer Protocol} (protocollo di trasferimento file), in informatica e nelle telecomunicazioni, è un protocollo per la trasmissione di dati tra macchine basato su TCP (ing. \textit{Transmission Control Protocol}, protocollo di rete responsabile di rendere affidabile la comunicazione dati tra mittente e destinatario) e con architettura di tipo client-server.\\
Il protocollo usa connessioni TCP distinte per trasferire i dati e per controllare i trasferimenti e richiede autenticazione del client tramite nome utente e password}
}

\newglossaryentry{MAC Address}
{
	name = {MAC Address},
	text = {MAC Address},
	sort = {mac address},
	description = {In informatica e telecomunicazioni il MAC Address(indirizzo MAC, dove MAC sta per Media Access Control) è un codice di 48 bit assegnato in modo univoco ad ogni scheda di rete Ethernet o Wireless prodotta al mondo}
}

\newglossaryentry{HTTPS}
{
	name= {HTTPS},
	text ={HTTPS},
	sort = {https},
	description= {Protocollo largamente utilizzato su Internet per la comunicazione sicura attraverso una rete di computer. HTTPS è il risultato dell'applicazione di un protocollo di crittografia asimmetrica al protocollo di trasferimento di ipertesti HTTP. È utilizzato per garantire trasferimenti riservati di dati nel web, in modo da impedire intercettazioni dei contenuti}
}

\newglossaryentry{W3C}
{
	name={W3C},
	text={W3C},
	sort={w3c},
	description={Il World Wide Web Consortium, anche conosciuto come W3C, è un'organizzazione internazionale, non governativa, che stabilisce standard tecnici per il World Wide Web inerenti sia i linguaggi di markup che i protocolli di comunicazione, al fine di assicurare la crescita del Web}
}	

\newglossaryentry{XML Schema}
{
	name={XML Schema},
	text={XML Schema},
	sort={xml schema},
	description={Lo XML Schema è un linguaggio di descrizione del contenuto di un file XML. Il suo scopo è delineare quali elementi sono permessi, quali tipi di dati sono ad essi associati e quale relazione gerarchica hanno fra loro gli elementi contenuti in un file XML. Lo XML Schema permette la convalida del file XML e l'estrazione da un file XML di un insieme di oggetti}
}
	
\newglossaryentry{SHA-256}
{
	name={SHA-256},
	text={SHA-256},
	sort={sha-256},
	description={SHA-256 è una funzione crittografica di Hash, ossia un algoritmo che trasforma dati di lunghezza arbitraria in una stringa di lunghezza fissa, da cui è impossibile risalire al messaggio originale. SHA-256 fa parte della famiglia di funzioni SHA-2 (\textit{Secure Hash Algoritm 2}) sviluppata dalla \textit{National Security Agency}. Il 256 che appare nel nome è la lunghezza, in bit, della stringa risultato}
}

\newglossaryentry{AES}
{
	name={AES},
	text={AES},
	sort={aes},
	description={AES, \textit{Advanced Encryption Standard}, conosciuto anche come \textit{Rijndael}, è un algoritmo di cifratura a blocchi utilizzato come standard dal governo degli Stati Uniti d'America. In base all'implementazione, la dimensione della chiave può essere di 128, 192 o 256 bit, mentre la dimensione dei blocchi è sempre di 128 bit. L'algoritmo sfrutta una sequenza di permutazioni e sostituzioni per cifrare un messaggio e in seguito poterlo decifrare}
}

\newglossaryentry{thread}
{
	name={Thread},
	text={Thread},
	sort={thread},
	description={È la più piccola sequenza di istruzioni programmate che possono essere gestite indipendentemente da uno \textit{scheduler} (componente del sistema operativo che coordina l'accesso alle risorse). Solitamente un thread è un componente di un processo, più thread possono esistere all'interno di un processo ed utilizzare le sue stesse risorse, possibilità non permessa ad un processo diverso}
}

\newglossaryentry{KeyContainer}
{
	name={KeyContainer},
	text={KeyContainer},
	sort=keycontainer,
	description={Contenitore di chiavi messo a disposizione dalle librerie di classi .NET Framework, il quale permette di salvare una coppia di chiavi asimmetriche in sicurezza, agevolandone la gestione}
}

\newglossaryentry{RSA}
{
	name={RSA},
	text={RSA},
	sort={rsa},
	description={Algoritmo di crittografia asimmetrica, inventato nel 1977 da Ronald Rivest, Adi Shamir e Leonard Adleman (le iniziali dei cognomi formano il nome dell'algoritmo) utilizzabile per cifrare o firmare informazioni.\\
Il sistema di crittografia si basa sull'esistenza di due chiavi distinte, dette comunemente pubblica e privata, che vengono usate per cifrare e decifrare. Se la prima chiave viene usata per la cifratura, la seconda deve necessariamente essere utilizzata per la decifratura e viceversa. Sebbene le due chiavi siano fra loro dipendenti, non è possibile risalire dall'una all'altra, garantendo l'integrità del metodo. Il poter decifrare un messaggio solo grazie alla chiave collegata a quella utilizzata per la cifratura permette di utilizzare l'algoritmo per firmare digitalmente dei dati}
}

\newglossaryentry{Reverse Engineering}
{
	name={Reverse Engineering},
	text={Reverse Engineering},
	sort={reverse engineering},
	description={\textit{Reverse engineering}, o ingegneria inversa, è il processo di estrazione di informazioni da un prodotto finito per poterlo riprodurre o costruire un nuovo prodotto sulla base delle informazioni ottenute. Spesso esso si traduce nel disassemblare un Software per poterne analizzare il funzionamento e riscriverlo a proprio piacimento}
}

\newglossaryentry{XML}
{
	name={XML},
	text={XML},
	sort={xml},
	description={XML, \textit{eXtensible Markup Language}, è un linguaggio di \textit{markup} (o marcatore) che definisce un insieme di regole per scrivere e controllare gli elementi contenuti in un documento in un formato leggibile sia dagli umani sia dai computer, utilizzando dei \textit{tag} (o etichette) che possono essere creati e definiti dagli utenti. Il linguaggio è estensibile ed è di facile interpretazione, perciò è largamente utilizzato nella trasmissione di documenti nel Web}
}

\newglossaryentry{HTTP}
{
	name={HTTP},
	text={HTTP},
	sort={http},
	description={\textit{HyperText Transfer Protocol}, o protocollo di trasferimento di un ipertesto, è un protocollo a livello applicativo usato come principale sistema per la trasmissione d'informazioni sul Web}
}

\newglossaryentry{Hash}
{
	name={Hash},
	text={Hash},
	sort={hash},
	description={Con il termine Hash si intende il risultato di una funzione crittografica di Hash, ossia un algoritmo matematico che trasforma dati di lunghezza arbitraria in una stringa di lunghezza fissa, detta appunto valore di Hash o più semplicemente Hash. Una funzione crittografica di Hash ha la caratteristica di non fornire mai (o con probabilità molto basse) due risultati uguali per stringhe di input diverse}
}